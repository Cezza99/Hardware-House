\documentclass[10pt]{article}
\usepackage[top=3cm, bottom=2cm, right=2cm, left=2cm]{geometry}
\usepackage{fancyhdr}
\usepackage{helvet}
\usepackage{multirow}
\usepackage{color, colortbl}
\usepackage[table, dvipsnames]{xcolor}
\usepackage{amssymb}
\usepackage{multicol}
\usepackage{enumitem}
\usepackage{graphicx}
\usepackage{listings}
\usepackage{tabularx}
%\usepackage[parfill]{parskip}

%\setlength{\columnseprule}{1pt} -------> Per inserire separatori fra le colonne

\newcommand{\centered}[1]{\begin{tabular}{l} #1 \end{tabular}}

\definecolor{LightCyan}{rgb}{0.88,1,1}
\definecolor{Gray}{gray}{0.96}

\renewcommand\familydefault{\sfdefault} 
\setlength{\parskip}{1em}

\pagestyle{fancy}
\fancyhf{}
\lhead{Hardware House}
\cfoot{Pagina \thepage}

\graphicspath{{imgs/}}

\title{\huge\textbf{HH: Hardware House}}
\author{Cesare Omodei, Francesco De Marchi}
\date{AA: 2019/2020}
